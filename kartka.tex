\documentclass[10pt,twocolumn]{article}
%\usepackage{showframe} %rysuje linie gdzie margines
\usepackage[a4paper,margin=1.15cm, top=1cm,bottom=1.25cm]{geometry}
\usepackage[T1]{fontenc}
\usepackage[polish]{babel}
\usepackage[utf8]{inputenc}
\usepackage{lmodern}
\selectlanguage{polish}
\usepackage{amsfonts} %\mathbb
\usepackage{sectsty} %\sectionfont
\usepackage{amsmath} %\underset matrix 

\sectionfont{\large}
\subsectionfont{\normalsize}

\begin{document}
\begin{flushleft}
\section{Uwarunkowanie}
$\delta+1 = \frac{1}{1-\delta} \qquad \delta^{2} \approx 0$\\
Błąd względny wyniku: $$\left| \frac{f(\widetilde{x}) - f(x)}{f(x)} \right| = \left| \frac{xf'(x)}{x} \right| |\delta|$$
$$cond(x) = \left| \frac{xf'(x)}{x} \right| $$
Uwarunkowanie zadania numerycznego: $$\frac{||f(\widetilde{d}) - f(d)||}{||f(d)||} \leq cond(d) \frac{||d - \widetilde{d}||}{||d||}$$

\section{Normy}
\subsection{Wektorowe}
$$ ||x||_{2} = \sqrt{\sum_{i=1}^{n} x_{i}^{2}} \ \ \ ||x||_{1} = \sum_{i=1}^{n} |x_{i}|^{2} $$
$$ ||x||_{\infty} = \max_{i \in \{1,...,n\}} |x_{k}| \ \ \  ||x||_{p} = \left(\sum_{i=1}^{n}|x_{i}|^{p}\right)^{\frac{1}{p}} $$
\subsection{Macierzowe}
$$ ||A||_{1} = \max_{j=1,...,n} \sum_{i=1}^{n}|a_{ij}| \ \ \ ||A||_{\infty} = \max_{i=1,...,n} \sum_{j=1}^{n}|a_{ij}| $$
$$\mathbb{}$$
$$ ||A||_{2} = \sup_{x \neq 0, \ x \in \mathbb{C}^{n}} \frac{||Ax||_{2}}{||x||_{2}} = \sqrt{\rho\left(A^{*}A\right)} \ \ \ A^{*} = \overline{A}^{T}$$
$$ ||A||_{F} = \sqrt{\sum_{i = 0}^{n} \sum_{j = 0}^{n} |a_{ij}|^{2}} $$
zgodność norm jeśli: $||Ax|| \leq ||A|| \cdot ||x||$\\
dla normy Frobeniusa: $||Ax||_{F} \leq ||A||_{F} ||x||_{2}$\\
dla dowolnej zachodzi: $||AB|| \leq ||A|| \cdot ||B||$

\section{Arytmetyka zmianno przecinkowa (fl)}
Zbiór $M( 2,t,k )$ nie jest zamknięty ze względu na działania arytmetyczne. 
$fl(x \diamond y) = rd(x \diamond y)$ zatem błąd arytmetyki jest taki sam jak błąd arytmetyki reprezentacji wyniku.\\
Zatem $fl(x \diamond y) = (x \diamond y)(1 + \delta) $\\ jeżeli $x = m_{1}\cdot2^{c_{1}}$ $y = m_{2}\cdot2^{c_{2}}$ oraz $c_{1} -c_{2} > t$ to $fl(x+y) = x$

\section{Nymerycza poprawność}
Def. Algorytm $A$ dla zadania $\varphi$ nazywamy numerycznie poprawnym jeśłi istnieje stała $k$ niezależna od wskaźnika uwarunkowania i niezależna od arytmetyki tż dla dowolnej danej $d \in D$ istnieje dana $\widetilde{d}$ tż $||d-\widetilde{d}|| \leq K\cdot 2^{-t}||d||$ oraz $fl(A(d)) = \varphi(\widetilde{d})$\\
Czyli, wynik algorytmu A dla danej d (dokładniej) w arytmetyce $fl$ jest dokładnym wynikiem zadania $\varphi$ dla nieco zaburzonej danej.\\
Oszacowanie błędu alg. num. poprawnego:\\
$||fl(A(d)) - \varphi(d)|| \leq cond(d) \frac{K\cdot2^{-t}||d||}{||d||}||\varphi(d)||$

\section{Interpolacja}
\subsection{Lagrange}
$$p_{n}(x) = \sum_{i = 0}^{n}f_{i}l_{i}(x)$$
gdzie: $$l_i(x) = \prod_{{j = 0},\ {j \neq i}}^{n} \frac{x - x_{j}}{x_{i}-x_{j}}$$
\subsection{Newton}
$$c_{0} + c_{1}(x-x_{0}) + c_{2}(x-x_{0})(x-x_{1}) + \ldots + c_{n}(x-x_{0})\ldots (x-x_{n-1})$$
gdzie: $$c_{k} = f_{0,1,2,\ldots,k}$$ oraz $$f_{i,i+1,\ldots,k+i} \stackrel{def}{=} \frac{f_{i+1,\ldots,k} + f_{i,\ldots,(i+k-1)}}{x_{i+k}-x_{i}}$$
Hermite identycznie tylko że w tabeli węzły się powtarzają i w miejscu różnic dzielonych których nie można otrzymać wpisujemy $\frac{f^{(k)}(x_i)}{k!}$ a w wielomianie interpolacyjnym składniki postaci $(x-x_i)$ będą posiadały odpowiednią potęgę

\section{Całkowanie numeryczne}
Kwadratura jest rzędy $r$ jeśli jest dokładna dla wszystkich wielomianów stopnie $r-1$ oraz istnieje wielomian stopnia r dla której nie jest dokładna\\
\subsection{Trapezy}
$$S(f) = \sum_{k=1}^{N} \frac{x_k-x_{k-1}}{2}{(f(x_k)+f(x_{k-1}})) =$$ $$=\frac{H}{2} \left(f(a)+f(b)+2\sum_{k=1}^{N-1}f(a+kH)\right)$$
$$|E(f)| = \left| \sum_{k=1}^{N}\frac{H^3f''(\xi_k)}{12} \right| \leq \frac{H^2(b-a)}{12}\underset{\xi \in [a,b]}{\sup}|f''(\xi)|$$
\subsection{Prostokąty}
$$S(f) = \sum_{k=1}^{N}(x_k-x_{k-1})f\left(\frac{x_k+x_{k-1}}{2}\right) = H\sum_{k=0}^{n}f\left(\frac{x_k + x_{k-1}}{2}\right)$$
$$|E(f)| \leq \frac{NH^3}{24}\underset{\xi \in [a,b]}{\sup}|f''(\xi)| = \frac{H^2}{24}(b-a)\underset{\xi \in [a,b]}{\sup}|f^{(4)}(\xi)|$$
\subsection{Simpson}
$$S(f) = \sum_{k=1}^N \frac{H}{6} \left(f(x_{k-1})+4f\left(x_{k-1}+\frac{H}{2}\right)+f(x_k)\right)$$
$$|E(f)|= \left| \sum_{k=1}^N \frac{H^5f^{(4)}(\xi_k)}{90\cdot2^5}\right| \leq \frac{H^4(b-a)}{2^4}\underset{\xi \in [a,b]}{\sup}|f^{(4)}(\xi)| $$

\section{Rozwiązywanie układów r. liniowych}
\subsection{Metoda Eliminacji Gaussa GE}
Tw.1 Jeśli $A$ jest macierzą dodatnio określoną, to metoda GE zastosowana do $Ax=b$ jest wykonalna\\
Tw.2 Jeśli A jest silnie diagonalnie dominująca, to metoda GE zastosowana do układu $Ax=b$ jest wykonalna
\subsection{GEPP}
W $k$-tym kroku wybieramy wierszowo (analog. kolumnowo) $\underset{i \in \{k,\ldots,n\}}{\max}|a_{ik}|$ i zamieniamy $k$-ty wiersz z wierszem z maksymalnym el. w macierzy $[A^{(k-1)}|b^{(k-1)}]$
\subsection{GECP}
w $k$-tym kroku znajdujemy $p,q$ tż. $|a_{pq}| = \underset{i,j \in \{1,\ldots,n\}}{\max}|a_{ij}|$ i zamieniamy $p$ wiersz z $k$ oraz $q$ kolumnę z $k$ przy zamianie kolumn następuje zamiana zmiennych w x
\subsection{Rozkład LU} %nie wiem czy warto coś pisać
\subsection{Rozkład PA=LU} 
\subsection{Banachiewicz-Cholesky}
$$A = LL^T =
\begin{bmatrix}
    l_{11} & 0 & 0 & \dots  & 0 \\
    l_{21} & l_{22} & 0 & \dots  & 0 \\
    \vdots & \vdots & \vdots & \ddots & \vdots \\
    l_{n1} & l_{n2} & l_{n3} & \dots  & l_{nn}
\end{bmatrix}
\begin{bmatrix}
    l_{11} & l_{21} & l_{31} & \dots  & l_{n1} \\
    0 & l_{22} & l_{32} & \dots & l_{n2} \\
    \vdots & \vdots & \vdots & \ddots & \vdots \\
    0 & 0 & 0 & \dots  & l_{nn} 
\end{bmatrix}
$$
Fakt: $\sqrt{\omega_1} = l_{11} \ldots \sqrt{\frac{\omega_{k}}{\omega_{k-1}}} = l_{kk}$ 

\section{Iteracyjne metody rozwiązywania u.r.l.}
Tw. Metoda iteracyjna $x^{(k+1)}=Bx^k+c$ jest zbieżna globalnie $\Leftrightarrow \rho(A) < 1$\\
Tw. Jeśli $||B|| < 1$ gdzie $||\cdot||$ jest normą zgodną z pewną normą wektorową to metoda $x^{(k+1)}=Bx^k+c$ jest zbieżna globalnie\\


\end{flushleft}
\end{document}

%tabelka do harmitta ale za dużo miejsca więc porzucona
\begin{center}
\begin{tabular}{|l|l|c|l|}
\hline
$x_0$ & $f(x_0)$ & & \\
\hline
$x_0$ & $f\prime(x_0)$ &  &\\
\hline
$x_0$ & $f\prime (x_0)$ & $\frac{f\prime \prime(x_0)}{2!}$ & \\
\hline
\ldots & & & \\
\hline
$x_n$ & $f_n$ & \ldots & $f_{0,1,\ldots,n}$\\
\hline
\end{tabular}
\end{center}
